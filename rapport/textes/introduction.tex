À la question \textit{« Pourquoi réaliser un correcteur orthographique en C ? »} nous répondons tous en coeur : « parce que on nous l'a demandé !». D'après Wikipedia, voici la définition d'un correcteur orthographique :
\begin{quote}
	« Un correcteur orthographique est, en informatique, un outil logiciel permettant d'analyser un texte afin de détecter, et éventuellement de corriger, les fautes d'orthographe et les coquilles qu'il contient. »
\end{quote}
\vspace{20px}
Le but de ce projet d'algorithmique est de réaliser un correcteur orthographique, capable de détecter des erreurs de langue à l'aide d'un dictionnaire et de proposer des corrections orthographiques pertinentes, puisées dans le dictionnaire donné.\\

Aujourd'hui tout le monde utilise au quotidien un correcteur orthographique, celui-ci étant le plus souvent déjà présent dans les smartphones et ordinateurs. Tout le monde s'est déjà fait sauver par un programme qui corrige une coquille de frappe lors de l'envoi d'un SMS ou d'un e-mail important. Un logiciel de correction orthographique ne peut pas remplacer un académicien chevronné mais permet de corriger une bonne partie des fautes d'orthographe que nous commettons, plus ou moins involontairement.\\

Pour nous aider dans la réalisation de ce projet, nous avons à notre disposition un dictionnaire de la langue française contenant plus de 300 000 mots. Ce dictionnaire doit pouvoir nous aider à proposer des corrections pertinentes pour une phrase donnée (contenant potentiellement des fautes) en argument à notre programme. Notre programme a pour objectif d'être proche du programme \textit{Aspell} du projet GNU.\\

Nous commencerons tout d'abord ce rapport par la définition des TAD utilisés ainsi que par l'analyse descendante correspondante au projet. Nous enchaînerons par la suite par la conception préliminaire puis détaillée du problème donné. Enfin nous terminerons par l'ensemble du code C de notre projet ainsi que les différents tests unitaires effectués.