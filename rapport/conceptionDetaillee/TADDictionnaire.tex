
\begin{algorithme}
	\begin{enregistrement}{Dictionnaire}
		\champEnregistrement{racines}{\tableauUneDimension{1...NBLETTRES}{de }{\elementdico}}
		\champEnregistrement{longueur}{\naturel}
	\end{enregistrement}
	\newline
	\fonction{dictionnaire}{}{\dictionnaire}
	{i : \naturelnonnul, dictionnaire : \dictionnaire}
	{
		\pour{i}{1}{NBLETTRES}{}
		{
			\affecter{dictionnaire.racines[i]}{\pointeurNULL}
		}
		\affecter{dictionnaire.longueur}{0}
		\retourner{dictionnaire}
	}
	\newline
	\fonction{estVide}{dictionnaire: \dictionnaire}{\booleen}
	{}
	{
		\retourner{dictionnaire.longueur = 0}
	}
	\newline
	\procedure{insererMot}{\paramEntreeSortie{dictionnaire: \dictionnaire}; \paramEntree{mot : \mot}}
	{
		i, j, indice : \naturelnonnul \\ 
		longueur : \naturel \\ 
		tableauDesElementsSuivants : \tableauUneDimension{1...NBLETTRES}{de }{\elementdico} \\ 
		element, nouvelElement : \elementdico
	}
	{
		\pour{indice}{1}{NBLETTRES}{}
		{
			\affecter{tableauDesElementsSuivants[indice]}{dictionnaire.racines[indice]}
		}
		\affecter{longueur}{dictionnaire.longueur}

		\pour{i}{1}{longueur(mot)}{}
		{
			\affecter{j}{1}
			\commentaire{On vient chercher dans notre tableau le caractère correspondant au caractère de notre mot}
			\tantque{j \infegalea longueur ET obtenirLettre(tableauDesElementsSuivants[j]) \differentde iemeCaractere(mot, i)}
			{
				\affecter{j}{j + 1}
			}
			\sialorssinon{j \infegalea longueur}
			{
				\affecter{element}{tableauDesElementsSuivants[j]}
			}
			{
				\affecter{nouvelElement}{element(iemeCaractere(mot, i), faux)}
				\commentaire{La racine n'existe pas dans le dictionnaire, on doit la créer}
				\sialors{i = 1}
				{
					\affecter{dictionnaire.racines[dictionnaire.longueur]}{nouvelElement}
					\affecter{dictionnaire.longueur}{dictionnaire.longueur + 1}
				}
				{
					\instruction{ajouterElementSuivant(element, nouvelElement)}
				}
				\affecter{element}{nouvelElement}
			}
			\commentaire{On passe à l'élément suivant pour le prochain caractère du mot}
			\pour{indice}{1}{NBLETTRES}{}
			{
				\affecter{tableauDesElementsSuivants[indice]}{(obtenirTableauDesElementsSuivants(element))[indice]}
			}
			\affecter{longueur}{obtenirLongueurTableauDesElementsSuivants(element)}
		}
		\instruction{ajouterPresence(element)}
	}
	\newline
	\procedure{insererFichier}{\paramEntreeSortie{dictionnaire: \dictionnaire}; \paramEntree{nomFichier : \chaine}}
	{
		fichier : \fichierTexte, \\
		ligne : \chaine 
	}
	{
		\affecter{fichier}{fichierTexte(nomFichier)}
		\instruction{ouvrir(fichier, lecture)}
		\pourChaque{ligne}{fichier}
		{
			\instruction{insererMot(dictionnaire, genererMot(ligne))}
		}
		\instruction{fermer(fichier)}
	}
	\newline
	\fonction{estPresent}{dictionnaire: \dictionnaire, mot : \mot}{\booleen}
	{
		i, j, indice : \naturelnonnul \\ 
		longueur : \naturel \\ 
		tableauDesElementsSuivants : \tableauUneDimension{1...NBLETTRES}{de }{\elementdico} \\ 
		element : \elementdico
	}
	{
		\pour{indice}{1}{NBLETTRES}{}
		{
			\affecter{tableauDesElementsSuivants[indice]}{dictionnaire.racines[indice]}
		}
		\affecter{longueur}{dictionnaire.longueur}
		\commentaire{Vérifions que l'on trouve le mot caractère après caractère dans le dictionnaire}
		\pour{i}{1}{longueur(mot)}{}
		{
			\affecter{j}{1}
			\tantque{j \infegalea longueur ET obtenirLettre(tableauDesElementsSuivants[j]) \differentde iemeCaractere(mot, i)}
			{
				\affecter{j}{j + 1}
			}
			\commentaire{On vérfie si on est sorti du while parce qu'on est arrivé à la fin du tableau ou si on a trouvé la bonne lettre}
			\sialorssinon{j \infegalea longueur ET obtenirLettre(tableauDesElementsSuivants[j]) = iemeCaractere(mot, i)}
			{
				\affecter{element}{tableauDesElementsSuivants[j]}
			}
			{
				\retourner{faux}
			}
			\commentaire{Tout va bien, on continue à avancer pour le prochain caractère}
			\pour{indice}{1}{NBLETTRES}{}
			{
				\affecter{tableauDesElementsSuivants[indice]}{(obtenirTableauDesElementsSuivants(element))[indice]}
			}
			\affecter{longueur}{obtenirLongueurTableauDesElementsSuivants(element)}
		}
		\commentaire{Si on n'a eu toujours aucune erreur, on retourne la valeur du booléen du dernier caractère}
		\retourner{obtenirBooleen(element)}
	}
	\newline
	\fonction{charger}{nomFichier : \chaine}{\dictionnaire}
	{
		fichier : \fichierTexte \\
		dictionnaire : \dictionnaire \\
		longueur : \naturel \\
		i : \naturelnonnul
	}
	{
		\affecter{dictionnaire}{dictionnaire()}
		\affecter{fichier}{fichierTexte(nomFichier)}
		\instruction{ouvrir(fichier, lecture)}
		\affecter{longueur}{recupererUneLongueur(fichier)}
		\affecter{dictionnaire.longueur}{longueur}
		\pour{i}{1}{longueur}{}
		{
			\affecter{dictionnaire.racines[i]}{chargerElementsSuivants(fichier)}
		}
		\instruction{fermer(fichier)}
		\retourner{dictionnaire}
	}
	\newline
	\fonction{chargerElementsSuivants}{fichier : \fichierTexte}{\elementDico}
	{
		caractere : \caractere \\
		booleen : \booleen \\
		longueur : \naturel \\
		element : \elementDico \\
		i : \naturelnonnul
	}
	{
		\affecter{caractere}{recupererUnCaractere(fichier)}
		\affecter{booleen}{recupererUnBooleen(fichier)}
		\affecter{longueur}{recupererUneLongueur(fichier)}
		\affecter{element}{element(caractere, booleen)}
		\pour{i}{1}{longueur}{}
		{
			\instruction{ajouterElementSuivant(element, chargerElementsSuivants(fichier))}
		}
		\retourner{element}
	}
	\newline
	\fonction{sauvegarder}{dictionnaire: \dictionnaire, emplacement : \chaine}{\fichierTexte}
	{
		fichier : \fichierTexte \\
		dizaine, unite : \caractere \\
		i : \naturelnonnul
	}
	{
		\affecter{fichier}{fichierTexte(emplacement)}
		\instruction{ouvrir(fichier, ecriture)}
		\instruction{naturelEnCaracteres(dictionnaire.longueur, dizaine, unite)}
		\sialors{dizaine \differentde '0'}
		{
			\instruction{ecrireCaractere(fichier, dizaine)}
		}
		\instruction{ecrireCaractere(fichier, unite)}
		\pour{i}{1}{dictionnaire.longueur}{}
		{
			\instruction{ecrireCaractere(fichier, obtenirLettre(dictionnaire.racines[i]))}
			\instruction{ecrireCaractere(fichier, booleenEnCaractere(obtenirBooleen(dictionnaire.racines[i])))}
			\instruction{naturelEnCaracteres(obtenirLongueurTableauDesElementsSuivants(dictionnaire.racines[i]), dizaine, unite)}
			\sialors{dizaine \differentde '0'}
			{
				\instruction{ecrireCaractere(fichier, dizaine)}
			}
			\instruction{ecrireCaractere(fichier, unite)}
			\instruction{sauvegarderElementsSuivants(fichier, dictionnaire.racines[i])}
		}
		\instruction{fermer(fichier)}
		\retourner{fichier}
	}
	\newline
	\procedure{sauvegarderElementsSuivants}{\paramEntreeSortie{fichier: \fichierTexte}; \paramEntree{elementDico : \elementDico}}
	{
		tableauElementsSuivants : \tableauUneDimension{1...NBLETTRES}{de }{\elementdico} \\
		dizaine, unite : \caractere \\
		i, indice : \naturelnonnul
	}
	{
		\pour{indice}{1}{NBLETTRES}{}
			{
				\affecter{tableauDesElementsSuivants[indice]}{(obtenirTableauDesElementsSuivants(elementDico))[indice]}
			}
		\pour{i}{1}{obtenirLongueurTableauDesElementsSuivants(elementDico)}{}
		{
			\instruction{ecrireCaractere(fichier, obtenirLettre(tableauElementsSuivants[i]))}
			\instruction{ecrireCaractere(fichier, booleenEnCaractere(obtenirBooleen(tableauElementsSuivants[i])))}
			\instruction{naturelEnCaracteres(obtenirLongueurTableauDesElementsSuivants(tableauElementsSuivants[i]), dizaine, unite)}
			\sialors{dizaine \differentde '0'}
			{
				\instruction{ecrireCaractere(fichier, dizaine)}
			}
			\instruction{ecrireCaractere(fichier, unite)}
			\instruction{sauvegarderElementsSuivants(fichier, tableauElementsSuivants[i])}
		}
	}
	\newline
\end{algorithme}
