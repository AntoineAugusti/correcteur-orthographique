\begin{algorithme}
	\procedure{asispell}{\paramEntree{argc : \naturel, argv : \tableauUneDimension{1..nbParametres}{ de }{\chaine}}}
	{
		p : \parametres
	}
	{
		\affecter{p}{formaterParametres(argc, argv)}
		\instruction{executerProgramme(p)}
	}
	\newline

	\fonction{formaterParametres}{nbParametre : \entier, parametres : \tableauUneDimension{1..nbParametres}{ de }{\chaine}}{\parametres}
	{
		optch : \entier \\
		format : \chaine \\
		p : \parametres
	}
	{
		\commentaire{un paramère simple : la lettre}
		\commentaire{un paramètre avec argument : la lettre suivi de ":"}
		\affecter{format}{"hd:f:"}


		\affecter{p}{parametres()}
		\commentaire{fonction équivalente au getopt() en C}
		\tantque{\affecter{optch}{parseLesParametres(argc, argv, format)} $\ne$ -1}
		{
			\sialors{optch = 'h'}
			{
				\instruction{fixerAide(p, vrai)}
			}
			\sialors{optch = 'f'}
			{
				\instruction{fixerFichier(p, optarg)}
			}
			\sialors{optch = 'd'}
			{
				\instruction{fixerDictionnaire(p, optarg)}
			}

		}
		
		\retourner{parametres}
	}
	\newline

	\procedure{executerProgramme}{\paramEntree{parametres : \parametres}}
	{
		d : \dictionnaire \\
		entreeStandard : \chaine
	}
	{
		\affecter{d}{dictionnaire()}
		\sialorssinon{obtenirAide(parametres)}
		{
			\instruction{afficherAide()}
		}
		{
			\commentaire{Si on a un fichier et un dictionnaire, on complète le dictionnaire déjà sérialisé}
			\sialorssinon{(obtenirFichier(parametres) \differentde NULL) ET (obtenirDictionnaire(parametres) \differentde NULL)}
			{
				\affecter{d}{charger(obtenirDictionnaire(parametres))}
				\instruction{insererFichier(d, obtenirFichier(parametres))}
				\instruction{sauvegarder(d, obtenirDictionnaire(parametres))}
			}
			{
				\affecter{entreeStandard}{lire()}
				\instruction{fixerCible(parametres, entreeStandard)}

				\commentaire{On effectue la correction avec le dictionnaire déjà sérialisé}
				\sialors{(obtenirCible(parametres) \differentde NULL) ET (obtenirDictionnaire(parametres)) \differentde NULL}
				{
					\affecter{d}{charger(obtenirDictionnaire(parametres))}
					\instruction{corriger(obtenirCible(parametres), d)}
				}
			}

		}
	}
	\newline

	\procedure{ajouterDansLeDictionnaire}{\paramEntreeSortie{dictionnaire : \dictionnaire}; \paramEntree{cheminDuFichier :\chaine}}
	{}
	{
		\instruction{insererFichier(dictionnaire, cheminDuFichier)}
	}
	\newline

	\fonction{corriger}{chaine : \chaine, dictionnaire : \dictionnaire}{\chaine}
	{
		chaineValide : \booleen \\
		erreur : \entier \\
		listeDeSuperMot : \listesupermot \\
		resultat : \chaine
	}
	{
		\instruction{evaluerSaisie(chaineValide, erreur, chaine)}
		\affecter{resultat}{""}
		\sialors{chaineValide}
		{
			\affecter{listeSuperMot}{genererListeDeSuperMots(chaine, dictionnaire)}
			\instruction{corrigerListeDeSuperMots(listeDeSuperMots, dictionnaire)}
			\instruction{afficherCorrection(listeDeSuperMots)}
		}
		\instruction(afficherResultat(erreur, chaine))
	}
	\newline

	\procedure{evaluerSaisie}{\paramEntreeSortie{estValide : \booleen, erreur : \entier};\paramEntree{chaine : \chaine}}
	{
		i : \naturelnonnul \\
	}
	{
		\affecter{i}{1}
		\affecter{\pointeur{estValide}}{1}
		\tantque{i \infegalea longueur(chaine) ET \pointeur{estValide}}
		{
			\sialors{non(estCaractereValide(iemeCaractere(chaine, i))}
			{
				\affecter{\pointeur{estValide}}{0}
				\affecter{\pointeur{erreur}}{1}
			}
			\affecter{i}{i + 1}
		}
	}
	\newline

	\fonction{genererListeDeSuperMots}{chaine : \chaine, dictionnaire : \dictionnaire}{\listesupermot}
	{
	}
	{
		\retourner{separerMots(chaine, dictionnaire)}

	}
	\newline

	\fonction{separerMots}{chaine : \chaine, dictionnaire : \dictionnaire}{\listesupermot}
	{
		i : \naturelnonnul \\
		listeRetour : \listesupermot \\
		motTemp : \mot \\
		superMotTemp : \supermot 
	}
	{
		\affecter{motTemp}{mot()}
		\affecter{listeRetour}{liste()}
	
		\pour{i}{1}{longueur(chaine) + 1}{}
			{
				\sialorssinon{estLettreOuTiret(iemeCaractere(chaine, i))}
					{
						\instruction{insererCaractere(motTemp, iemeCaractere(chaine, i), longueur(motTemp)+1)}
					}
					{
						\sialors{longueur(motTemp) $\ne$ 0}
							{
								\instruction{fixerMot(superMotTemp, motTemp)}
								\instruction{fixerPosition(superMotTemp,i - (longueur(motTemp) + 1))}
								\instruction{fixerValidite(superMotTemp, verifieValidite(motTemp, dictionnaire))}
								\instruction{inserer(listeRetour, 1, superMotTemp)}
							}

						\affecter{motTemp}{mot()}
					}
			}
		\retourner{listeRetour}
	}
	\newline

	\fonction{verifieValidite}{mot : \mot, dictionnaire \dictionnaire}{\booleen}
	{
	}
	{
		\retourner{estPresent(dictionnaire, mot)}
	}
	\newline

	\procedure{corrigerListeSuperMot}{\paramEntreeSortie{listeDeSuperMots : \listesupermot}; \paramEntree{dictionnaire : \dictionnaire}}
	{
		i : \entier
		superMotTemp : \supermot
	}
	{
		\pour{i}{1}{longueur(listeDeSuperMots)}{}
		{
			\affecter{superMotTemp}{obtenirElement(listeDeSuperMots, i)}
			\sialors{non superMotTemp.estValide}
			{
				\affecter{superMotTemp.listeDeCorrections}{proposerMot(superMotTemp.mot, dictionnaire)} 
			}
		}
	}
	\newline

	\fonction{afficherCorrection}{listeDeSuperMots : \listesupermot}{\chaine}
	{
		i, j : \entier \\
		superMotTemp : \supermot \\
		chaine : \chaine
	}
	{
		\affecter{chaine}{""}
		\pour{i}{1}{longueur(listeDeSuperMots)}{}
		{
			\affecter{superMotTemp}{obtenirElement(listeDeSuperMots, i)}
			\sialorssinon{superMotTemp.estValide = VRAI}
			{
				\affecter{chaine}{chaine + "*"}
			}
			{

				\affecter{chaine}{chaine + "\&\_ " + obtenirContenu(superMotTemp.mot) + "\_" }
				\affecter{chaine}{chaine + entierEnChaine(longueur(superMotTemp.listeCorrections)) + "\_" }
				\affecter{chaine}{chaine + entierEnChaine(superMotTemp.position) + ":\_" }
				\affecter{chaine}{chaine + listeDeMotsEnChaine(superMotTemp.listeCorrections)}
			}
			\affecter{chaine}{chaine + \textit{retout à la ligne}}
		}
		\instruction{ecrire(chaine)}
	}
	\newline

	\fonction{listeDeMotsEnChaine}{listeDeMots : \listemot}{\chaine}
	{
		i : \entier \\
		chaine : \chaine \\
		motTemp : \mot
	}
	{
		\affecter{chaine}{""}
		\pour{j}{1}{longueur(listeDeMots)}{}
		{
			\affecter{motTemp}{obtenirElement(listeDeMots, j)}
			\affecter{chaine}{chaine + "\_" + obtenirContenu(motTemp)}
		}
		\retourner{chaine}
	}
	\newline

	\procedure{afficherResultat}{\paramEntree{erreur : \entier, chaine : \chaine}}
	{}
	{
		\sialorssinon{erreur = 0}
		{
			\instruction{afficherCorrection(chaine)}
		}
		{
			\instruction{afficherErreur(erreur)}
		}
	}
	\newline
	
	\procedure{afficherErreur}{\paramEntree{typerDErreur : \entier}}
	{}
	{
			\sialorssinon{typerDErreur = 1}
			{
				\ecrire{Au moins un caractère entré n'est pas valide.}
			}
			{
				\sialors{typerDErreur = 2}
				{
					\ecrire{Type d'erreur 2}
				}
			}
	}
	\newline

	\procedure{afficherCorrection}{\paramEntree{correction : \chaine}}
	{}
	{
		\ecrire{correction}
	}
	\newline

	\procedure{afficherAide}{}
	{}
	{
		\ecrire{'Aide asispell :'}
		\ecrire{'		asispell -h : cette aide'}
		\ecrire{'		asispell -d dico : correction de l'entrée standard à l'aide du dictionnaire dico}
		\ecrire{'		asispell -d dico - fic : compléter le dictionnaire dico à l'aide du fichier fic}
	}
	\newline
\end{algorithme}