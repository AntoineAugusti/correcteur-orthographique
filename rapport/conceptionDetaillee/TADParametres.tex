\begin{algorithme}

	\commentaire{Structure permettant de manipuler les paramètres reçus par le programme (-h, -d, -f, -malandain etc etc ..)}
	\begin{enregistrement}{Parametre}
		\commentaire{variable initialisée si il y a un '-d' dans les paramètres, elle contient le nom du dictionnaire}
		\champEnregistrement{dictionnaire}{\chaine}

		\commentaire{variable initialisée si il y a un '-f' dans les paramètres, elle contient le nom du fichier}
		\champEnregistrement{fichier}{\chaine}

		\commentaire{La chaîne de caractères donnée en entrée à corriger}
		\champEnregistrement{cible}{\chaine}

		\commentaire{Variables qui prend vrai SSI il y a un '-h' dans les paramètres.}
		\champEnregistrement{aide}{\booleen}
	\end{enregistrement}	


	\fonction{parametres}{}{\parametres}
	{
		p : \parametres
	}
	{
		\affecter{p.dictionnaire}{NULL}
		\affecter{p.fichier}{NULL}
		\affecter{p.cible}{NULL}
		\affecter{p.aide}{Faux}
		\instruction{}
	}
	\newline
	\procedure{fixerDictionnaire}{\paramEntreeSortie{p : \parametres}, \paramEntree{c : \chaine}}
	{}
	{
		\affecter{p.dictionnaire}{c}
	}
	\newline
	\procedure{fixerFichier}{\paramEntreeSortie{p : \parametres}, \paramEntree{c : \chaine}}
	{}
	{
		\affecter{p.fichier}{c}
	}
	\newline
	\procedure{fixerCible}{\paramEntreeSortie{p : \parametres}, \paramEntree{c : \chaine}}
	{}
	{
		\affecter{p.cible}{c}
	}
	\newline
	\procedure{fixerAide}{\paramEntreeSortie{p : \parametres}, \paramEntree{b : \booleen}}
	{}
	{
		\affecter{p.aide}{b}
	}
	\newline
	\fonction{obtenirDictionnaire}{p : \parametres}{\chaine}
	{}
	{
		\retourner{p.dictionnaire}
	}
	\newline
	\fonction{obtenirFichier}{p : \parametres}{\chaine}
	{}
	{
		\retourner{p.fichier}
	}
	\newline
	\fonction{obtenirAide}{p : \parametres}{\booleen}
	{}
	{
		\retourner{p.aide}
	}
	\newline
	\fonction{obtenirCible}{p : \parametres}{\chaine}
	{}
	{
		\retourner{p.cible}
	}
	\newline
\end{algorithme}